Ganz tolle Auswertung des Versuchs. \cite{Dem10}

und noch eine super Grafik

%\begin{figure}[h]
%	\centering\includegraphics[width=0.8\textwidth]{fig/figure_1}
%	\caption{Extrem coole Grafik!}
%\end{figure}
%
%\begin{figure}[h]
%	\centering\includegraphics[width=0.8\textwidth]{fig/figure_2}
%	\caption{Extrem coole Grafik!}
%\end{figure}
%
%\begin{figure}[h]
%	\centering\includegraphics[width=0.8\textwidth]{fig/figure_3}
%	\caption{Extrem coole Grafik!}
%\end{figure}
%
%\begin{figure}[h]
%	\centering\includegraphics[width=0.8\textwidth]{fig/figure_4}
%	\caption{Extrem coole Grafik!}
%\end{figure}


% SUBFIGURE
%\begin{figure}[h]
%	\centering
%	\subfigure[]{\includegraphics[width=0.49\textwidth]{fig/2_a_plot}}
%	\subfigure[]{\includegraphics[width=0.49\textwidth]{fig/2_b_plot}}
%	\caption{\textbf{a)} Temperatur des Aluminium Hohlzylinders gegen die Zeit aufgetragen, \textbf{b)} Änderung der Temperatur gegenüber der Temperatur aufgetragen. Jeweils unter Verwendung des Heizdrahtes.}
%	\label{fig:2_1_plot}
%\end{figure}

%\xtable{htb}{amplituden}{Amplitudenverhältnisse der schweren und leichten Gleiter in der zweiatomigen Kette}{(Aufg. 1)}

%Flo

%Anselm
