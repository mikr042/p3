\section{Erwartungswert}
Modell $m$ einer Messung besteht aus wahrem Wert $w$ und einem zufälligen Beitrag $z$. Eine Messreihe ist eine Stichprobe aus der Gesamtheit
\begin{align}
		m = w + z
\end{align}

$z$ kann mehrere Bestandteile haben. Einen zufälligen Beitrag ("Rauschen"), Einfluss der Genauigkeit des verwendeten Messgeräts und es kann eine theoretische Unsicherheit existieren. Was nicht passieren sollte ist ein Fehler im Messprozess.
\begin{itemize}
	\item statistische Unsicherheit
	\item systematische Unsicherheit
	\item theoretische Unsicherheit
\end{itemize}

Einfachste Schätzer ist das arithmetische Mittel
\begin{align}
	\bar{x} = \frac{1}{n} \sum_{l=1}^{n}x_{l}
\end{align}

Dabei geht eine Messunsicherheit einher
\begin{align}
	\bar{x} \pm u_{x}
\end{align} 