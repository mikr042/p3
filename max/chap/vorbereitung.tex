In diesem Versuch wird die Materialanalyse mit Hilfe von Röntgenstrahlung kennengelernt. Es handelt sich dabei um eine Technik mit der Eigenschaften von Materialien wie zum Beispiel die Schichtdicke oder die Zusammensetzung bestimmt werden können. Daher findet sie unter anderem in der Materialwissenschaft und in der Industrie bei der Qualitätskontrolle Anwendung.
Des Weiteren wird diese Technik auch bei der Gepäckkontrolle an Flughäfen angewendet.

\section{Erzeugung von Röntgenstrahlung}

Bei Röntgenstrahlung handelt es sich um elektromeagnetische Strahlung im Wellenlängenbereich $\lambda=\SI{250}{\pico\metre}$ bis $\SI{1}{\femto\metre}$. Es gibt zwei Methoden diese zu erzeugen, die charakteristische Strahlung und die Bremsstrahlung.

\subsection{Charakteristische Röntgenstrahlung}

Die charakteristische Strahlung entsteht bei Abregung von Elektronen aus höheren Schalen in schweren Atomen, wie zum Beispiel Eisen oder Wolfram. Dazu wird das Material mit schnellen Elektronen beschossen.
Die emittierte Strahlung besitzt diskrete Frequenzen gemäß
\begin{equation}
 \hbar\omega = E_{m}-E_{n}.
\end{equation}
