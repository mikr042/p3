%\begin{figure}[h]
%	\centering\includegraphics[width=0.8\textwidth]{fig/figure_4}
%	\caption{Extrem coole Grafik!}
%\end{figure}


% SUBFIGURE
%\begin{figure}[h]
%	\centering
%	\subfigure[]{\includegraphics[width=0.49\textwidth]{fig/2_a_plot}}
%	\subfigure[]{\includegraphics[width=0.49\textwidth]{fig/2_b_plot}}
%	\caption{\textbf{a)} Temperatur des Aluminium Hohlzylinders gegen die Zeit aufgetragen, \textbf{b)} Änderung der Temperatur gegenüber der Temperatur aufgetragen. Jeweils unter Verwendung des Heizdrahtes.}
%	\label{fig:2_1_plot}
%\end{figure}

\section{Kalibrierung des R\"ontgendetektors}

Der versuch wird wie in der Vorbereitung beschrieben durchgef\"uhrt.
Dabei konnte kein Nutzen durch die Verwendung einer h\"oheren Verst\"arkungsstufe beobachtet werden, die K$_\alpha$ und K$_\beta$-Linien der Standardproben lie\ss en sich auch bei V = 4 nicht klar voneinander unterscheiden. Die Kalibrierung wird daher auch nur anhand der K$_\alpha$-Linien durchgef\"uhrt.\\
Einzige Ausnahme bildet Silber, bei dem im R\"ontgenspektrum zwei klar erkennbare Peaks vorhanden war. Diese lagen allerdings in einem so hohen Energiebereich, dass sie mit einer Verst\"arkung von V = 4 nicht mehr auf der Skala gewesen w\"are.\\
Die gefitteten Peaks der einzelnen Proben befinden sich in Tabelle \ref{tab:a1_peaks}.

\xtable

